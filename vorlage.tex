\documentclass[a4paper, 12pt]{article}

\usepackage[ngerman]{babel}
\usepackage[latin1]{inputenc}
%\usepackage{graphicx}
\usepackage{natbib}
\usepackage{times}
\usepackage[left=30mm,right=40mm,top=25mm,bottom=20mm, includeheadfoot, centering]{geometry}
\usepackage{parskip}
\usepackage{setspace}
\usepackage[center]{caption2}
\usepackage[flushmargin,bottom,hang]{footmisc}
\usepackage{acronym}

\pagestyle{myheadings}
\renewcommand{\baselinestretch}{1.3}
\renewcommand{\arraystretch}{1.0}
\setlength{\parindent}{0pt}
\setlength{\parskip}{0cm}
\setlength{\footnotemargin}{1em}

\renewcommand{\labelenumi}{(\alph{enumi})}
\renewcommand{\labelenumii}{(\alph{enumi}\arabic{enumii})}



\begin{document}


\begin{titlepage}

\renewcommand{\thefootnote}{\fnsymbol{footnote}}


\begin{center}
\noindent {\Large Hinweise zur Anfertigung einer Bachelor-/ Diplom-/Master-/
Seminar-/Studienarbeit}

\vspace{10ex} {\Huge Bachelor-/Diplom-/Master-/ Seminar-/Studienarbeit
\footnote{Entsprechendes ausw\"{a}hlen}}


\vspace{10ex}


  \noindent eingereicht bei \\
  \vspace{10ex}
  bei Prof. Dr. Werauchimmer\\
Institut f\"{u}r Wirtschaftswissenschaft\\
Abteilung f\"{u}r Wasauchimmer\\
Fakult\"{a}t f\"{u}r Energie- und Wirtschaftswissenschaften\\ Technische Universit\"{a}t
Clausthal\\[10ex] von \\ Wer auch immer\\
Julius-Albert-Str. 2\\
38678 Clausthal-Zellerfeld\\ Telefon: 05323-727633 \\ Studienrichtung:
Wirtschaftswissenschaften\\ 24. Fachsemester \\ Matrikelnummer: 0815007 \\[2ex]
Datum: \today

\end{center}



\end{titlepage}
%Inhaltsverzeichnis soll r\"{o}misch numeriert sein:
\pagenumbering{roman}

%Erstellt das Inhaltsverzeichnis:
\addcontentsline{toc}{section}{Inhaltsverzeichnis}\tableofcontents
\renewcommand{\baselinestretch}{1.5}
\pagebreak

%Erstellt das Abk\"{u}rzungsverzeichnis:
\section*{Abk\"{u}rzungsverzeichnis}
\emph{Abbildungs-, Abk\"{u}rzungs- und Tabellenverzeichnisse k\"{o}nnen bei nur wenigen
Eintr\"{a}gen entfallen.} \vspace{0.5em}
\addcontentsline{toc}{section}{Abk\"{u}rzungsverzeichnis}%

\singlespacing

\begin{acronym}[XXXXX]
\acro{TUC}{Technische Universit\"{a}t Clausthal}
\end{acronym}

%Erstellt das Tabellenverzeichnis:
\listoftables \addcontentsline{toc}{section}{Tabellenverzeichnis}
\renewcommand{\baselinestretch}{1.5}
 \pagebreak


%Der Rest soll arabisch numeriert sein:
\pagenumbering{arabic}\setcounter{page}{1}

\clearpage \setcounter{footnote}{0}
\setlength{\parskip}{6pt}

\sloppy
\renewcommand{\thefootnote}{\arabic{footnote}}

\section{Einleitung}

Bachelor-, Diplom-, Master-, Seminar- und Studienarbeiten stellen
Pr\"{u}fungs\-leistungen dar, welche Studierende am Institut f\"{u}r Wirtschaftswissenschaft
der Technischen Universit\"{a}t Clausthal (\acs{TUC}) im Rahmen ihres Studiums zu
erbringen haben. An die Anfertigung dieser wissenschaftlichen Arbeiten werden
zahlreiche wissenschaftlich-methodische und formal-technische Anforderungen
gestellt.

Durch die Anfertigung einer wissenschaftlichen Arbeit sollen die Studierenden
zeigen, dass sie in der Lage sind, ein \"{o}konomisches Problem selbstst\"{a}ndig und
innerhalb einer bestimmten Frist unter der Anwendung einschl\"{a}giger
wissenschaftlicher Methoden zu bearbeiten. Das Anfertigen einer derartigen Arbeit
stellt neben der eigentlichen inhaltlichen Aufgabenstellung h\"{a}ufig auch eine
schwierige formal-technische Aufgabe dar.

Zu der Frage \glqq Wie man eine wissenschaftliche Abschlu\ss arbeit schreibt\grqq\
(Eco, 1993) existieren eine ganze Reihe von Ratgebern, welche Grundkenntnisse und
praktische Empfehlungen f\"{u}r die Anfertigung wissenschaftlicher Arbeiten
liefern.\footnote{Vgl. z.B. \cite{bri2005}, \cite{rop2005}, \cite{fra2004},
\cite{dis1998}, \cite{bae1996}, \cite{cor1996}, \cite{eco1993} oder \cite{the1993}.
Die vorliegenden Ausf\"{u}hrungen basieren in Teilen auf der Arbeit von \cite{rid2001},
mit welcher der Autor zu seinen Studienzeiten gute Erfahrungen gemacht hat.} Ferner
bietet das Institut f\"{u}r Wirtschaftswissenschaft auch eine eigens zu diesem Thema
konzipierte Lehrveranstaltung an, welche allen Studierenden des Instituts f\"{u}r
Wirtschaftswissenschaft offen steht.

Das Ziel dieser vorliegenden Arbeit stellt die Vermittlung eines \"{U}berblicks \"{u}ber die
wesentlichen Grundregeln zur Erstellung wissenschaftlicher Arbeiten dar. Ferner
liefert die Arbeit einige Vorschl\"{a}ge zur Gestaltung formaler Inhaltselemente. Diese
\"{U}berblicksarbeit kann zudem als Beispielvorlage f\"{u}r eine eigene wissenschaftliche
Arbeiten an der TU Clausthal verwendet werden. Auf Grund der Vielzahl z.B. m\"{o}glicher
Zitiertechniken kann sie allerdings keinen Anspruch auf Vollst\"{a}ndigkeit erheben.
Insbesondere die detaillierte formale Gestaltung von wissenschaftlichen Arbeiten
sollte stets mit den zust\"{a}ndigen Betreuerinnen und Betreuern abgestimmt werden.

In den folgenden Kapiteln \ref{Verfassen} und \ref{Formales} werden
wissenschaftsmethodische und formale gestalterische Anforderungen an diese Arbeiten
n\"{a}her dargestellt. Im abschlie\ss enden Kapitel \ref{Beratung} werden die wesentlichen
Ergebnisse zu einem Bewertungskanon zusammengef\"{u}hrt.

\section{Das Verfassen einer wissenschaftlichen Abschlussarbeit}\label{Verfassen}

Die wesentlichen Bewertungskriterien f\"{u}r wissenschaftliche Arbeiten stellen der
Aufbau der eigenen (!) Argumentation sowie die Analyse der f\"{u}r die Themenstellung
relevanten Literatur dar. Diese beiden Aspekte werden in den nachfolgenden
Ausf\"{u}hrungen eingehender erl\"{a}utert. Im folgenden Abschnitt \ref{AAufbau} wird
zun\"{a}chst das methodische Vorgehen n\"{a}her erl\"{a}utert, w\"{a}hrend der Abschnitt
\ref{Umgang} die Recherche und Auswertung der themenrelevanten Literatur zum
Gegenstand hat.

\subsection{Der Aufbau einer wissenschaftlichen Arbeit} \label{AAufbau}
\subsubsection{Fragestellung und Vorgehensweise}

Den Kern einer jeden wissenschaftlichen Arbeit stellt die Entwicklung einer
Fragestellung dar, auf welcher alle folgenden Bearbeitungsschritte aufzubauen sind.
Je pr\"{a}ziser dieser grundlegende erste Bearbeitungsschritt vollzogen wird, desto eher
kann die grundlegende Zielsetzung der wissenschaftlichen Arbeit abgeleitet und die
eigene Argumentation darauf ausgerichtet werden. F\"{u}r die Ableitung dieser
Fragestellung ist i.d.R. eine erste Durchsicht der einschl\"{a}gigen Literatur
notwendig, um dadurch \"{u}berhaupt erst die relevanten Themen und Problemfelder
erschlie\ss en und als Fragestellung auch formulieren zu k\"{o}nnen (z.B.: Welche
wettbewerblichen Wirkungen k\"{o}nnen von nat\"{u}rlichen Monopole in Netzindustrien
ausgehen?). Aus dieser Problemerschlie\ss ung l\"{a}sst sich somit die Zielsetzung der
Arbeit ableiten, und dieses stellt bereits ein wesentliches Teilergebnis der
wissenschaftlichen Arbeit dar (z.B.: Ziel der vorliegenden Arbeit ist es, die
wettbewerblichen Wirkungen von nat\"{u}rlichen Monopolen in Netzindustrien auf den
Eintritt neuer Anbieter im Markt zu analysieren.)

Nur aus einer entwickelten Fragestellung heraus ergibt sich \"{u}berhaupt erst die
M\"{o}glichkeit die f\"{u}r die eigene Arbeit relevante Literatur zu ordnen. So stellt die
Aufarbeitung und Zusammenf\"{u}hrung bestehender Erkenntnisse inklusive einer kritischen
Diskussion den Grundbaustein der Bearbeitung der Fragestellung dar. Unter dem Ordnen
von Literatur wird dabei die Herstellung eines ausreichend begr\"{u}ndeten Zusammenhangs
zwischen der eigenen Zielsetzung sowie den Darstellungen und Argumentationen bereits
vorhandener Arbeiten hergestellt. Die Zielsetzung fungiert somit quasi als der
Beginn  des \glqq roten Fadens\grqq, welcher die erarbeiteten Erkenntnisse
problembezogen miteinander verkn\"{u}pft und auf die Zielsetzung der Arbeit bezieht. Die
Strukturierung des zu bearbeitenden \"{o}konomischen Problems dr\"{u}ckt die
Eigenst\"{a}ndigkeit der Argumentationsf\"{u}hrung aus und \"{u}bt damit einen ma\ss geblichen
Einfluss auf die Qualit\"{a}t der wissenschaftlichen Arbeit aus.

Von besonderer Bedeutung f\"{u}r diese Strukturierungsleistung sind die Einleitung und
die Ergebnisteile einer Arbeit. Sie bilden die \glqq logische\grqq\ Klammer f\"{u}r die
Bearbeitung der Hauptteile. Die vorliegenden Ausf\"{u}hrungen haben zum Ziel, die in der
Untersuchung verwendete Systematik ausreichend darzulegen und die auf die
Zielsetzung bezogenen Ergebnisse der Arbeit herauszustellen.

Eine wichtige Aufgabe der Einleitung stellt die inhaltliche Begr\"{u}ndung der gew\"{a}hlten
Vorgehensweise dar: So findet eine Einf\"{u}hrung in die Problemstellung der Arbeit
statt, an welcher die Relevanz der vorliegenden Fragestellung inhaltlich abgeleitet
und das Analyseziel  explizit formuliert wird. Basierend auf diesen einf\"{u}hrenden
\"{U}berlegungen wird der Gang der Argumentation entwickelt und die Verbindungen zu den
einzelnen Bearbeitungsschritten, d.h. zu den einzelnen  Unterkapiteln der Arbeit,
hergestellt. Die Einleitung kann zudem auch dazu verwendet werden, um bei eher
allgemein gehaltenen Problemstellungen (z.B. Weiterentwicklungen des Menschenbildes
des homo oeconomicus) relevante Grundbegriffe zu kl\"{a}ren als auch Abgrenzungen des
Themas (z.B. Ergebnisse experimenteller Untersuchungen) vorzunehmen.

Das Ziel einer jeden wissenschaftlichen Arbeit stellt stets die Erarbeitung neuer
Erkenntnisse dar. F\"{u}r den Bereich von Seminar-, Studien-, Diplom-, Bachelor- und
Masterarbeiten sind dabei in erster Linie Schlussfolgerungen, welche sich aus der
eigenen Literaturanalyse ergeben, auf die Zielsetzung der Arbeit zu beziehen. Diese
Schlussfolgerungen k\"{o}nnen sich zum einen auf die jeweiligen Bearbeitungsschritte
(also Kapitel) der Arbeit beziehen und werden als Teilergebnisse der Analyse in
einem entsprechenden Zwischenfazit am Ende des jeweiligen Kapitels zusammengefasst.
Zum anderen sind dies allerdings auch Schlussfolgerungen, welche sich aus der
Synthese und kritischen Diskussion der jeweiligen Teilergebnisse im abschlie\ss enden
Ergebnisteil ergeben. Schwerpunkte einer solchen Ergebnisdiskussion k\"{o}nnen die
Zusammenfassung der wesentlichen Ergebnisse, aber auch die Formulierung
weiterf\"{u}hrender Thesen und/oder offener Fragen sein.

Die Konzeption und Umsetzung dieser logischen \glqq Klammer\grqq\ der Argumentation
- ausgehend von der Zielsetzung \"{u}ber die einzelnen Begr\"{u}ndungsschritte bis zu den
Ergebnissen der Arbeit - ist kein einzelner, in sich abgeschlossener Arbeitsschritt,
sondern ein die gesamte Bearbeitung begleitender Prozess. Das mehrfache Entwerfen
und kontinuierliche Pr\"{u}fen der eigenen Argumentation bringt diesen Reifungsprozess
einer wissenschaftlichen Arbeit zum Ausdruck.

\subsubsection{Die Gliederungssystematik}\label{Gliederungssystematik}

Die Gliederung der Arbeit ist das formale Gegenst\"{u}ck zur Argumentationsf\"{u}hrung. Sie
gibt eine erste Information \"{u}ber den Inhalt und die Vorgehensweise einer
wissenschaftlichen Arbeit. Gliederungs\"{u}berschriften, die den zentralen Inhalt der
jeweiligen Bearbeitung in einer aussagekr\"{a}ftigen Form wiedergeben, sind daher f\"{u}r
das Verst\"{a}ndnis der Argumentationsf\"{u}hrung von herausgehobener Bedeutung. Die
relevanten Themen- oder Gedankengruppen werden durch die Aufgliederung in Teile,
Kapitel oder Abschnitte inhaltlich abgegrenzt und - dem Bearbeitungsgang
entsprechend - in ihrer Reihenfolge angeordnet.

Eine folgerichtige und in sich geschlossene Gedankenf\"{u}hrung zeigt sich in einem auch
formal einwandfreien Gliederungsaufbau. Positionen, die inhaltlich den gleichen Rang
einnehmen, stehen auch in der Gliederung auf derselben Stufe. Untergliederungen
bringen die relevanten Teilaspekte einer Problemstellung zum Ausdruck. Sie dienen
dem Zweck, verschiedene Aspekte der \"{u}bergeordneten gemeinsamen Problemstellung
auszuarbeiten. Einzelne Unterpunkte wiederholen daher nicht den \"{u}bergeordneten Punkt
wortgetreu, sondern weisen wesentliche Unterscheidungsmerkmale aus (vgl. Tabelle
\ref{TabGliederung}).

\begin{center}
\begin{table}[h] \centering
\renewcommand{\arraystretch}{1.3}
\begin{tabular}{|r|r|}
\multicolumn{1}{c}{\hspace{5cm}} & \multicolumn{1}{c}{\hspace{5cm}} \\ \hline
\multicolumn{1}{|c|}{\glqq falsch\grqq} & \multicolumn{1}{c|}{\glqq richtig\grqq} \\
\hline
3. Potenziale und Entwicklung & 3. Die Marktsituation von erneuerbaren  \\
erneuerbarer Energien & Energien \\ \hline
3.1 Potenziale & 3.1 Potenziale erneuerbarer Energien \\
\hline
3.2 Entwicklung & 3.2 Entwicklung der Stromerzeugung \\
& auf Basis erneuerbarer Energien \\
\hline
\end{tabular}
\singlespacing \caption{Problemorientierung im
Gliederungsaufbau}\label{TabGliederung} {Quelle: Eigene Darstellung}
\renewcommand{\arraystretch}{1.3}
\onehalfspacing
\end{table}
\end{center}

Zur Klassifikation der Gliederungspunkte stehen verschiedene Systematiken zur
Verf\"{u}gung. \"{U}bliche Klassifikationssysteme sind in Tabelle 2 zusammenfassend
dargestellt. Die zu bestimmende Tiefe der Gliederung h\"{a}ngt von Gegenstand, Art und
L\"{a}nge der Arbeit ab, wobei diese auch im Interesse der \"{U}bersichtlichkeit zu
beurteilen ist.

Da Unterabschnitte (z.B. 2.2.1, 2.2.2) jeweils einem \"{u}bergeordneten Problemkreis
(z.B. 2.2) zugeordnet sind, sollten diese Abschnitte - soweit dies m\"{o}glich und
sinnvoll ist - sowohl untereinander als auch im Vergleich mit anderen Abschnitten
(z.B. 4.1.1 bis 4.1.3) von gleichem Gewicht sein. Beim Untergliedern von Kapiteln
ist zudem zu beachten, dass auf Grund dieses Prinzips auf ein Unterkapitel (z.B.
1.1) immer mindestens ein weiteres Unterkapitel (z.B. 1.2) folgt. Bei der Gliederung
eines Textes sollte zudem darauf geachtet werden, dass der Text nicht unn\"{o}tigerweise
in zu viele Gliederungspunkte nahezu \glqq zergliedert\grqq\ wird, da hierdurch die
Lesbarkeit des Textes u.U. deutlich eingeschr\"{a}nkt werden kann. \glqq Br\"{u}che\grqq\ in
der Argumentationsf\"{u}hrung stellen einen gravierenden konzeptionellen Mangel
schriftlicher Arbeiten dar. Die kritische Pr\"{u}fung der Gliederungssystematik im
Hinblick auf die Klarheit und Nachvollziehbarkeit der Begr\"{u}ndungsschritte - die
\glqq Stringenz\grqq - ist daher auch f\"{u}r den Verfasser oder die Verfasserin eine
wichtige M\"{o}glichkeit, um Widerspr\"{u}che oder fehlende Verbindungen im Gedankengang
aufzudecken und die gew\"{a}hlte Vorgehensweise zu \"{u}berdenken.

\begin{center}
\begin{table}[h] \centering
\renewcommand{\arraystretch}{1.3}
\begin{tabular}{|l|l|}
\multicolumn{1}{c}{\hspace{5cm}} & \multicolumn{1}{c}{\hspace{5cm}} \\ \hline
\multicolumn{1}{|c|}{Numerische Gliederung} & \multicolumn{1}{c|}{Gemischte (alphanumerische)} \\
\multicolumn{1}{|c|}{} & \multicolumn{1}{|l|}{Klassifikation} \\
\hline 1 & A\\ \hline 2 & B \\ \hline 2.1 & 1 \\ \hline 2.2 & 2 \\ \hline 2.2.1 &
2.2 \\ \hline 2.2.2 & 2.2 \\ \hline 2.3 & 3 \\ \hline 3 & C \\ \hline
\end{tabular}
\singlespacing \caption{Klassifikationssysteme}\label{Klassifikation} {Quelle:
Eigene Darstellung}
\renewcommand{\arraystretch}{1.3}
\onehalfspacing
\end{table}
\end{center}


\subsection{Der Umgang mit wissenschaftlicher Literatur} \label{Umgang}

\subsubsection{Die Suche nach relevanterLiteratur}

Am Beginn einer jeden wissenschaftlichen Arbeit steht immer die Recherche und
Beschaffung der f\"{u}r die Themenstellung relevanten Literatur, um dadurch \"{u}berhaupt
erst eine inhaltliche Kompetenz bez\"{u}glich der verfolgten Themenstellung erlangen zu
k\"{o}nnen. Ein grundlegender Einstieg wird h\"{a}ufig durch die von den Betreuenden der
Arbeit ausgeh\"{a}ndigte Basisliteratur gew\"{a}hrt. Wie der Name schon verlauten l\"{a}sst,
stellt diese Literatur allerdings im Regelfall nur den Einstieg in die Bearbeitung
des Themas dar.

Neben dieser Basislekt\"{u}re er\"{o}ffnen h\"{a}ufig auch (Hand-)W\"{o}rterb\"{u}cher der Betriebs-
oder Volkswirtschaftslehre einen ersten Einstieg in das Thema, in welchen h\"{a}ufig
kurze Abhandlungen die Breite des bereits existierenden Fachwissens aufzeigen.
Daneben existieren zu spezifischeren Fragestellungen v.a. in internationalen
Journals h\"{a}ufig auch \"{U}berblicksartikel zu speziellen Forschungsprogrammen (sog.
\glqq Surveys\grqq). Ebenso vermitteln selbstverst\"{a}ndlich einschl\"{a}gige Lehrb\"{u}cher
themenspezifische \"{U}berblicke sowie Ansatzpunkte f\"{u}r die Recherche. Insgesamt sollte
sich mit Hilfe dieser genannten Quellen h\"{a}ufig ein problemloser Einstieg in das
jeweilige Themengebiet finden lassen.

F\"{u}r eine vertiefende Literaturrecherche ist schlie\ss lich eine systematische
Durchsicht der f\"{u}r das Thema relevanten nationalen und internationalen
Fachzeitschriften notwendig, in welchen aktuelle Forschungsergebnisse ver\"{o}ffentlich
werden. Diese Arbeiten lassen sich h\"{a}ufig bereits anhand des Themas oder der
Zusammenfassung (Abstract) auf ihre thematische Relevanz hin untersuchen. Bei der
Auswahl der Artikel sind insbesondere auch der thematische Schwerpunkt einer
Zeitschrift sowie deren Zielgruppen zu ber\"{u}cksichtigen. Ferner lassen sich in
Fachzeitschriften h\"{a}ufig auch wissenschaftliche Rezensionen zu weiteren aktuellen
Ver\"{o}ffentlichungen finden, welche ebenfalls f\"{u}r die eigene Literaturbeschaffung
herangezogen werden k\"{o}nnen.

Die zentralen Orte f\"{u}r die eigene Literatursuche stellen zun\"{a}chst die Bibliothek des
Instituts f\"{u}r Wirtschaftswissenschaft sowie die Universit\"{a}tsbibliothek dar. Beide
Einrichtungen verf\"{u}gen \"{u}ber unterschiedlichste Suchinstrumente (Kataloge,
Datenbanken etc.), anhand derer nach m\"{o}glicher Literatur recherchiert werden
kann.\footnote{Die Universit\"{a}tsbibliothek bietet regelm\"{a}\ss ig Einf\"{u}hrungskurse in die
Benutzung der Bibliothek an.} Ferner verf\"{u}gen im Netzwerk der TU Clausthal
angeschlossene Rechner \"{u}ber die M\"{o}glichkeit, in der elektronischen
Zeitschriftenbibliothek der TUC in zahlreichen (inter-) nationalen Journals online
zu recherchieren. Einige Journals werden in gedruckter Form auch in der
Institutsbibliothek Wirtschaftswissenschaft gehalten. Eine M\"{o}glichkeit zu einer
umfassenderen Recherche bietet die Fernleihe \"{u}ber den Gemeinsamen Bibliotheksverbund
(http://www.gbv.de). \"{U}ber das Internet lassen sich ferner auch spezifische Archive
mit aktuellen, noch nicht publizierten Forschungsarbeiten (sog. \glqq Working Paper
Archives\grqq) durchsuchen. Derartige Forschungsarbeiten stehen h\"{a}ufig auf den
Internetseiten von Forschungseinrichtungen zum freien Download zur Verf\"{u}gung. Die
Verwendung nichtwissenschaftlicher Quellen, wie etwa Abfragen aus
Internetsuchmaschinen oder Vorlesungsunterlagen, sollte in jeder Arbeit
grunds\"{a}tzlich unterbleiben. Je nach Art und v.a. Aktualit\"{a}t der Themenstellung l\"{a}sst
sich eine Verwendung nichtwissenschaftlicher Quellen jedoch h\"{a}ufig nicht vollst\"{a}ndig
ausschlie\ss en.

\subsubsection{Die Verarbeitung von Literaturquellen}

Der \glqq Literaturkasten\grqq\ kann durch die unterschiedlichen
Recherchem\"{o}glichkeiten schnell breit und reichhaltig gef\"{u}llt werden. Zwei Fragen
schlie\ss en sich an: Welche dieser Quellen sind f\"{u}r das Thema der Arbeit relevant und
welche sind f\"{u}r die Bearbeitung der eigenen Problemstellung geeignet? Zur
Beantwortung dieser Fragen ist eine kontinuierliche Auswertung sowie eine der
Zielsetzung angemessene Einordnung der einzelnen Literaturquellen
erforderlich.\footnote{Vgl. hierzu auch \cite{hae2000}.}

Die Bearbeitungsschritte erfordern ein tiefes inhaltliches Verst\"{a}ndnis sowie eine
Verarbeitung der einzelnen Quellen. Da die Vielfalt an Informationen, die Verwendung
einer fachspezifischen Sprache oder der einfache \glqq Glaube\grqq\ an die
Richtigkeit der getroffenen schriftlichen Ausf\"{u}hrungen oftmals eine kritische
Reflexion der Ausf\"{u}hrungen erschweren, sollten in der Vor- und Nachbereitung des
Textes die folgenden Grundregeln beachtet werden:\footnote{Vgl. \cite{koe1988}, S.
43f. Zum Lesen wissenschaftlicher Texte vgl. auch \cite{sch2003}.}

\begin{itemize}

\item Gewinnen eines \"{U}berblicks, um Zielsetzung, grundlegende Inhalte und den Aufbau
des Beitrags sowie die wissenschaftliche Denkrichtung zu verstehen;

\item Formulieren von Fragen, um das eigene Interesse am Beitrag zu pr\"{a}zisieren;

\item Erfassen der Inhalte, um Kernaussagen und wissenschaftliches Vorgehen (wie Art
der Aussagen, Methodik des empirischen Vorgehens) nachzuvollziehen;

\item Pr\"{u}fen des Erkl\"{a}rungsbeitrags der Aussagen, um etwa die Art des
wissenschaftlichen Erkenntnisinteresses (deskriptiv, normativ, funktional), das
methodische Vorgehen zur Hypothesenformulierung (deduktiv, induktiv) und auch eigene
Interpretationen des Autors zu unterscheiden und diese in den Themenkontext der
eigenen Arbeit einzuordnen;

\item Feststellen, welche Punkte offen geblieben sind, welche weiteren Fragen sich
anschlie\ss en bzw. welche Verbindungen sich zu den Erkenntnissen auch anderer Autoren
herstellen lassen.

\end{itemize}

Die Erstellung eines Auszugs (Exzerpt) des gelesenen Beitrags in eigenen Worten
zeigt nicht nur, ob der Text verstanden worden ist. Vielmehr dient ein Exzerpt auch
der Dokumentation, wenn z.B. entsprechende Stichworte gebildet werden. Zudem
erleichtert es die auf das eigene Thema bezogene Einordnung, wenn die spezifischen
Erkenntnisse mit den Aussagen anderer Beitr\"{a}ge verglichen werden.

Das Schreiben einer wissenschaftlichen Arbeit ist somit stets ein durchaus
dynamischer Prozess, der durch zahlreiche Wechselwirkungen zwischen den einzelnen
Arbeitsschritten gepr\"{a}gt ist. Dies schlie\ss t anf\"{a}ngliche \glqq Irrwege\grqq\ in der
Bearbeitung nicht aus und erfordert von daher immer eine kontinuierliche Pr\"{u}fung der
beabsichtigten Argumentationsf\"{u}hrung im Hinblick auf die Vollst\"{a}ndigkeit der
Themenerschlie\ss ung und die Nachvollziehbarkeit des Argumentationsaufbaus.
Erfahrungsgem\"{a}\ss  kann es als sinnvoll erachtet werden, Gliederungsentw\"{u}rfe mit den
Kommilitonen/-innen und mit dem Betreuer oder der Betreuerin des Themas zu
diskutieren.

\section{Der formale Rahmen der Anfertigung von wissenschaftlichen
Arbeiten}\label{Formales}

An die Anfertigung von wissenschaftlichen Arbeiten werden \"{a}u\ss erst spezielle
Anforderungen an die Textbearbeitung und die Textgestaltung gestellt. Der
wesentliche Grund f\"{u}r diese formalen Anforderungen an den Nachweis der verwendeten
Quellen ist, dass im Regelfall nicht eigene, sondern vielmehr das Gedankengut
anderer Autorinnen und Autoren den Inhalt der Literaturbearbeitung darstellt. Aus
diesen Grunde ergibt die Anforderung an wissenschaftliche Arbeiten, die verwendeten
Quellen in einer transparenten und nachvollziehbaren Art und Weise darzulegen (Kap.
\ref{Textbearbeitung}). Anforderungen an die Textgestaltung ergeben sich aus der
Notwendigkeit, aus pr\"{u}fungstechnischen Gr\"{u}nden f\"{u}r eine einheitliche und
vergleichbare Bearbeitung Sorge zu tragen (Kap. \ref{Textgestaltung}).

\subsection{Anforderungen an die Textbearbeitung} \label{Textbearbeitung}

\subsubsection{Die Quellenangabe im Text}

Der Leitgedanke beim Quellennachweis ist, einerseits die direkt oder indirekt
\"{u}bernommenen Gedanken eines anderen Autors kenntlich zu machen und andererseits dem
Leser oder der Leserin die M\"{o}glichkeit zu er\"{o}ffnen, anhand des Zitates den
aufgenommenen Gedanken zur\"{u}ckzuverfolgen. In diesem Sinne ist einwandfreies Zitieren
Ausdruck wissenschaftlicher Sorgfalt.

\"{U}bernommenes fremdes Gedankengut ist auch aus pr\"{u}fungsrechtlichen Erfordernissen
kenntlich zu machen. Wird fremdes Gedankengut verwendet und nicht als solches
gekennzeichnet, so stellt dieses einen Betrugsversuch dar: Die Arbeit ist dann mit
der Note \glqq 5\grqq\ zu bewerten. Bei Abgabe von Diplom-/Bachelor- oder
Masterarbeiten ist eine entsprechende schriftliche Erkl\"{a}rung zur Abfassung der
Arbeit abzugeben. Diese Erkl\"{a}rung ist bei allen abzugebenden Exemplaren original zu
unterschreiben. Ein Muster f\"{u}r diese ehrenw\"{o}rtliche Erkl\"{a}rung ist im Anhang 1
dargestellt. Bei einer Gruppenarbeit ist diese Erkl\"{a}rung von jedem Bearbeiter und
jeder Bearbeiterin f\"{u}r den von ihm/ihr verfassten Teil der Arbeit abzugeben.

Quellenangaben sind bei \glqq klassischer \grqq\ Zitation - ebenso wie sachliche
Randbemerkungen des Verfassers - in Fu\ss noten aufzunehmen.\footnote{Grunds\"{a}tzlich ist
auch eine Zitation im Harvard-Stil zul\"{a}ssig, bei welchem die Quelle im laufenden
Text in Klammern genannt wird. Siehe dazu das Eco-Zitat auf S. 1.} Fu\ss noten sind als
abgek\"{u}rzte S\"{a}tze aufzufassen. Die Quellenzuordnung erfolgt in der Regel durch eine
hochgestellte Zahl am Ende des Zitates, wobei unterhalb des Textes der jeweiligen
Seite in einer Fu\ss note die Quelle angegeben wird.

Fu\ss noten fangen immer mit Gro\ss buchstaben an und enden mit Punkten. Sie sollten am
Fu\ss e der jeweiligen Seite angegeben werden. Text und Fu\ss noten sind durch einen
Strich voneinander abzugrenzen. Die Fu\ss noten werden mit der gleichen Schriftart wie
im Text wiedergegeben. Die Schriftgr\"{o}\ss e sollte 2 pt kleiner als der Haupttext sein.
Die Nummerierung der Fu\ss note erfolgt seiten-, kapitel- oder textm\"{a}\ss ig. Die
Fu\ss notenziffer wird im Text hochgestellt und i. d. R. mit einer Schriftgr\"{o}\ss e kleiner
als der Haupttext formatiert. Wichtig ist die eindeutige Zuordnung der Fu\ss note zu
einem Satzteil, zu einem Satz oder zu einem Absatz.

Durch Zitate wird im Text auf den Zusammenhang mit dem Gedankengut anderer
hingewiesen. Die Verwendung von fremdem Gedankengut ist durch eine genaue
Quellenangabe deutlich in der Arbeit zu vermerken.\footnote{Zur Form von
Sekund\"{a}rbelegen vgl. z.B. \cite{bae1996}, S. 51.}  Zitate sind   wenn m\"{o}glich aus
der Prim\"{a}rquelle zu entnehmen und sollten das (und nur das) enthalten, was Sie mit
dem w\"{o}rtlichen oder sinngem\"{a}\ss en Zitat belegen m\"{o}chten. Nur wenn das Originalwerk
objektiv nicht zug\"{a}nglich ist, kann nach einer Quellenangabe in der
Sekund\"{a}rliteratur zitiert werden. Der Quellenhinweis gibt in diesem Fall mit dem
Hinweis \glqq Zitiert nach ... \grqq\ auch die Sekund\"{a}rliteratur an. Die
Quellenangaben sind ausreichend und eindeutig genug anzuf\"{u}hren, um die Quelle und
die angesprochene Stelle leicht wieder finden zu k\"{o}nnen. Bei der Verwendung von
Quellen aus dem Internet ist stets die genaue Adresse (URL) sowie das Datum des
Abrufs der Quelle mit anzugeben.

Man unterscheidet mehrere \textbf{Zitatformen}: Von einem \textbf{direkten Zitat}
(Zitat i. e. S.) wird gesprochen, wenn Ausf\"{u}hrungen von Dritten w\"{o}rtlich in den
eigenen Text \"{u}bernommen werden. Diese \"{U}bernahmen sind buchstaben- und zeichengetreu
vorzunehmen. Abweichungen vom Original sind deshalb durch eingeklammerte Zus\"{a}tze mit
einem Hinweis, z.B. \glqq Anm. d. Verf.\grqq, deutlich zu kennzeichnen.
Hervorhebungen (z.B. Unterstreichungen etc.) im zitierten Text sollten grunds\"{a}tzlich
\"{u}bernommen werden; eigene Hervorhebungen sind mit dem Zusatz \glqq Herv. durch
Verf.\grqq\ zu kennzeichnen. Auslassungen in einem Zitat sind durch mehrere Punkte
anzudeuten. Englische Zitate sind grunds\"{a}tzlich im Original zu zitieren, l\"{a}ngere
fremdsprachige Zitate sind zu \"{u}bersetzen und in einer Fu\ss note in der Originalsprache
anzugeben. Ein direktes Zitat wird im Text zwischen Anf\"{u}hrungsstriche gesetzt und in
der Fu\ss note ohne den Zusatz \glqq Vgl.\grqq\ belegt. W\"{o}rtliche Zitate sollten kurz
sein und eigene Formulierungen nicht ersetzen. Zitate sind angebracht, wenn es auf
die Demonstration der im Zitat gew\"{a}hlten Formulierung ankommt. Bei der Wiedergabe
l\"{a}ngerer Textpassagen ist eine entsprechende Hervorhebung zu empfehlen (z.B. durch
einzeiligen Abstand und/oder Texteinz\"{u}gen).

Bei einem \textbf{indirekten bzw. sinngem\"{a}\ss en Zitat} (Zitat i. w. S.) handelt es
sich um jede Form der inhaltlichen Anlehnung oder sinngem\"{a}\ss en Wiedergabe fremder
Gedanken und Ausf\"{u}hrungen in der eigenen Arbeit. Sie ist als solche anzugeben. Die
Quellen werden in der Fu\ss note mit einem \glqq vgl.\grqq\ bzw. \glqq Vgl.\grqq\
(=vergleiche) versehen.

Bei der \textbf{Zitiertechnik} ist der Vollbeleg oder der Kurzbeleg m\"{o}glich. Zu
Beginn der Arbeit sollte die Entscheidung f\"{u}r eine dieser Techniken getroffen werden
und diese gew\"{a}hlte Technik in der gesamten Arbeit auch konsequent angewendet werden.

Beim \textbf{Vollbeleg} sind diejenigen Daten zumindest beim Erstbezug auf die
betreffende Quelle vollst\"{a}ndig in Fu\ss noten anzugeben, so wie dies auch im
Literaturverzeichnis erfolgt. Zus\"{a}tzlich sind noch die Seite(n), die zitiert werden,
zu nennen. Wird sich mehrfach auf dieselbe Quelle bezogen, so k\"{o}nnen folgende
\textbf{Verk\"{u}rzungen} verwendet werden:

Wird die Quelle in \textbf{zwei aufeinander folgenden} Fu\ss noten genannt, so kann in
der zweiten Fu\ss note die Verk\"{u}rzung \glqq Ebenda\grqq\ verwendet werden. (Bsp.: FN 1:
Vgl. Kruschwitz, Lutz: Investitionsrechnung, 6. Aufl., Berlin, New York, 1995, S. 65
66. FN 2: Ebenda, S. 63.)

Wird die Quelle \textbf{mehrfach}, jedoch nicht aufeinander folgend in den Fu\ss noten
genannt, so kann bei der zweiten und jeder darauf folgenden Nennung der Quelle die
Verk\"{u}rzung \glqq a. a. O.\grqq\ verwendet werden. (Bsp.: FN 1: Vgl. Kaas, Klaus P.:
Informations\"{o}konomik, in: Tietz, Bruno/K\"{o}hler, Richard/Zentes, Joachim (Hrsg.),
\emph{Handw\"{o}rterbuch des Marketing}, 2. Aufl., Stuttgart, 1995, SP. 972. FN2: Vgl.
Kruschwitz, Lutz: Investitionsrechnung, 6. Aufl., Berlin, New York, 1995, S. 65 66.
FN 3: Vgl. Kaas, Klaus P.: Informations\"{o}konomik, a. a. O., SP. 975). Diese
Verk\"{u}rzung ist allerdings nur dann zul\"{a}ssig, wenn in der Arbeit keine weiteren
Quellen desselben Autors herangezogen wurden.

Beim Kurzbeleg werden nur wenige bibliographische Angaben in der jeweiligen Fu\ss note
ungeachtet ihrer erstmaligen oder wiederholten Nennung wiedergegeben, wobei es
verschiedene M\"{o}glichkeiten gibt: Name, Stichwort, Jahr, Zitatstelle (Bsp.: Vgl.
K\"{u}ting/Weber, Bilanzanalyse, 1997, S. 21.), oder Name, Stichwort, Zitatstelle (Bsp.:
L. Kruschwitz, Investitionsrechnung, S. 25.), oder Name, Jahr, Zitatstelle (Bsp.:
Baumol, 2000, S. 43.). Die Zeichensetzung wird unterschiedlich gehandhabt.
Variationen sind Anf\"{u}hrungsstriche f\"{u}r das Stichwort, keine Kommata,
Unterstreichung/Kursivdruck des Stichwortes oder des Namens. Geben Sie wenn m\"{o}glich
immer die genaue Seite (\glqq S. 100\grqq) bzw. Seiten (\glqq S. 100-101\grqq) an.

Um die vollst\"{a}ndige Quellenbezeichnung im Literaturverzeichnis zu finden, ist in die
Quellenangabe beim Kurzzitat der Name des Autors/der Autorin mit der in Klammern
hinzugef\"{u}gten Jahresangabe sowie die Seitenzahl aufzunehmen. Entsprechend erh\"{a}lt im
Literaturverzeichnis jede Quelle als \glqq Kennziffer\grqq\ das Erscheinungsjahr,
das hinter den Namen des Autors/der Autorin gesetzt wird. Bei mehr als zwei
Autoren/innen kann in der Quellenangabe die Abk\"{u}rzung et al. (Lat.: Abk\"{u}rzung f\"{u}r et
alii, alternativ: u.a. f\"{u}r \glqq und andere\grqq) verwendet werden. Beispiel: Bei
der Zitierweise Brockhoff, 1994, S. 25 ist der Kurzbeleg nicht ausreichend, wenn es
mehrere Quellen mit denselben Autoren und Jahr gibt. Es ist dann entweder der
Vorname bei mehreren Autoren mit gleichem Nachnamen oder ein Zusatz bei der
Jahreszahl (z. B. 1994a und 1994b) bei mehreren Quellen aus demselben Jahr
hinzuzuf\"{u}gen. Jedes Zitat muss zudem daraufhin \"{u}berpr\"{u}ft werden, ob es nicht - aus
dem Zusammenhang genommen - einen anderen als den ihm vom Autor oder der Autorin
gegebenen Sinn erh\"{a}lt.

\subsubsection{Das Literaturverzeichnis als Quellennachweis}

Im Literaturverzeichnis sind s\"{a}mtliche, der Abfassung der Arbeit zu Grunde liegenden
Quellen anzugeben. Die Quellen werden im Allgemeinen in alphabetischer Reihenfolge
nach Nachnamen der Verfasser geordnet bzw. als \glqq ohne Verfasser\grqq (o. V.)
eingeordnet. Auf eine einheitliche Form der bibliographischen Angaben ist innerhalb
der Arbeit zu achten.

Die einzelnen Angaben zur Bezeichnung der Quellen in den Quellenangaben und im
Literaturverzeichnis sind den Quellen selbst zu entnehmen. Dabei ist der Haupttitel,
nicht der Einbandtitel, zu nennen. Zus\"{a}tzliche Angaben k\"{o}nnen zweckm\"{a}\ss ig sein; hier
ist von Fall zu Fall zu entscheiden. Eine wichtige Orientierungsfunktion hat hierbei
die CIP-Einheitsaufnahme der Deutschen Bibliothek in Verbindung mit den Angaben zum
Urheberrecht. Folgende Angaben sind im einzelnen erforderlich:

\begin{enumerate}

\item bei B\"{u}chern/Monographien:

Name des Verfassers/der Verfasserin (oder der Autoren/-innen) und Initialen des/der
Vornamen(s) (akademische Grade und Titel werden nicht genannt), Erscheinungsjahr und
Titel des Werkes, Auflage (falls erforderlich) Verlagsort (bei mehr als drei
Verlagsorten ist der erste mit usw. zu nennen) Beispiel: Schwindt, C. (2005):
Resource Allocation in Project Management. Berlin.

\item bei Zeitschriftenaufs\"{a}tzen:

Name des Verfassers/der Verfasserin (oder der Autoren/-innen) und Initialen des/der
Vornamen(s) Erscheinungsjahr und Titel des Aufsatzes In: Titel der Zeitschrift
(Erscheinungsort bei wenig bekannten Zeitschriften) Jahrgang und Nummer des Heftes
und Seitenangabe mit der ersten und letzten Seiten- bzw. Spaltenzahl des Aufsatzes.
Beispiel: Zimmermann, J. und Schwindt, C. (2002): \emph{Parametrische Optimierung
als Instrument zur Bewertung von Investitionsprojekten}, In: Zeitschrift f\"{u}r
Betriebswirtschaft, Jahrgang 72, Heft 6, S. 593-617.

\item bei Sammelwerken:

Name des Verfassers/der Verfasserin (oder der Autoren/-innen) und Initialen des/der
Vornamen(s) , Erscheinungsjahr und Titel des Aufsatzes, In: Name und Initialen des
Vornamens der oder des Herausgebers, (Hrsg.), Titel, Verlagsort, Seitenzahlen,
Auflage (falls erforderlich), Verlagsort und Seitenangabe mit der ersten und letzten
letzten Seiten- bzw. Spaltenzahl des Aufsatzes. Beispiel: Erlei, M. und Jost, P.-J.
(2001): Theoretische Grundlagen des Transaktionskostenansatzes, In: Jost, P.-J.
(Hrsg.), \emph{Der Transaktionskostenansatz in der Betriebswirtschaftslehre},
Stuttgart, S. 35-75.

\end{enumerate}

Mehrere Ver\"{o}ffentlichungen eines Verfassers bzw. einer Verfasserin werden
chronologisch ansteigend entsprechend ihrem Erscheinungsjahr eingestellt. Mehrere
Ver\"{o}ffentlichungen eines Verfassers/einer Verfasserin aus einem Jahr werden mit
einem alphabetischen Index (a, b, ...) versehen, der entsprechend in den
Quellennachweis aufzunehmen ist. Beispiel: Schenk-Mathes, H. (2000a):
Kooperationsformen auf dem Energiemarkt unter besonderer Ber\"{u}cksichtigung des
Contracting, In: Beck, H.-P. u.a. (Hrsg.), \emph{Handbuch Energiemanagement},
Heidelberg. Schenk-Mathes, H. (2000b): \"{O}konomische Instrumente zur Steuerung von
umweltbezogenen Aktivit\"{a}ten in Unternehmen, In: Baur, J.F. (Hrsg.),
\emph{Umweltschutz und Energieversorgung im nationalen und internationalen
Rechtsrahmen}, K\"{o}ln, S. 9-24.

Bislang ist noch nicht abschlie\ss end gekl\"{a}rt, wie Quellen aus dem
Internet\footnote{Vgl. dazu \cite{alb1998}, S. 1368. Als Einf\"{u}hrung in die
Literaturrecherche im World Wide Web \cite{jss1996}.} zu zitieren sind, obwohl
gerade auch diese Form der Recherche in der Vergangenheit an Bedeutung gewonnen hat.
So stehen mittlerweile zahlreiche wissenschaftliche Arbeiten entweder exklusiv oder
in ihrer neuesten Fassung nur im Internet zur Verf\"{u}gung. F\"{u}r derartige Quellen
bietet sich ein der Zitierweise zu Fachzeitschriften analoges Vorgehen an, damit
auch diese Quellen ausreichend informativ sowie eindeutig nachzuvollziehen sind. Wie
bereits an anderer Stelle erw\"{a}hnt, ist f\"{u}r Internetquellen die Angabe der
vollst\"{a}ndigen URL sowie des Datums des Ausdrucks obligatorisch. Beispiel: Romstad,
E. (2000): Environmental Performance: An Extension of Weitzman's Pri-ces vs.
Quantities,
\emph{http://www.soc.uoc.gr/calendar/2000EAERE/papers/PDF/B5-Romstad.pdf},
05.12.2002.

\subsection{Anforderungen an die Textgestaltung} \label{Textgestaltung}

\subsubsection{Ordnungsschema, Umfang und Erscheinungsbild}

F\"{u}r die Gestaltung von wissenschaftlichen Arbeiten wird das in Tabelle 3
dargestellte Ordnungsschema empfohlen. Bei jeder Position ist zudem angegeben, ob
diese f\"{u}r Bachelor- (B), Diplom- (D), Master- (M) und Studienarbeiten (St) und/oder
Seminararbeiten (S) erforderlich ist. Au\ss er den leeren Deckbl\"{a}ttern und dem
Titelblatt sind s\"{a}mtliche Bl\"{a}tter zu nummerieren.\footnote{Zu Nummerierungsarten
vgl. Abschnitt \ref{Gliederungssystematik}.}

\begin{center}
\begin{table}[h] \centering
\renewcommand{\arraystretch}{1.3}
\begin{tabular}{|l|l|l|}
\multicolumn{1}{c}{\hspace{2cm}} & \multicolumn{1}{c}{\hspace{5cm}} &
\multicolumn{1}{c}{\hspace{5cm}}\\ \hline 1. & Leeres Deckblatt & St, D, B, M \\
\hline
2. & Titelblatt & S, St, D, B, M \\
\hline 3. & Inhaltsverzeichnis & S, St, D, B, M
\\
\hline 4. & (fach-/themenspezifisches) Abk\"{u}rzungsverzeichnis (bei Bedarf) & S, St,
D,
B, M \\
\hline
5. & Verzeichnis der Tabellen und Abbildungen (bei Bedarf) & S, St, D, B, M \\
\hline
6. & Haupttext & S, St, D, B, M \\
\hline
7. & Literaturverzeichnis & S, St, D, B, M \\
\hline
8. & Anhang (ggf. mit Verzeichnis) & St, D, B, M \\
\hline
9. & Ehrenw\"{o}rtliche Erkl\"{a}rung & St, D, B, M \\
\hline 10. & Leeres Deckblatt & St, D, B, M \\
\hline
\end{tabular}
\singlespacing \caption{Aufbau einer wissenschaftlichen Arbeit}\label{Aufbau}
{Quelle: Eigene Darstellung}
\renewcommand{\arraystretch}{1.3}
\onehalfspacing
\end{table}
\end{center}

Alle Inhaltselemente einer Arbeit vor dem eigentlichen Haupttext sollten in
r\"{o}mischen Ziffern nummeriert werden. Der Haupttext wird in arabischen Ziffern
beginnend mit 1 nummeriert. Die Seitenummerierung wird im Literaturverzeichnis und
im Anhang fortgesetzt.

Bachelor-, Diplom-, Master- und Studienarbeiten sind stets gebunden und mit einem
festen Umschlag verbunden abzugeben. Die Anzahl der einzureichenden Exemplare und
weitere formale Anforderungen sind im Pr\"{u}fungsamt oder ggf. bei den Betreuern der
Arbeit zu erfragen. Seminararbeiten sind i.d.R. in doppelter Ausf\"{u}hrung
anzufertigen. Sie k\"{o}nnen in einem Schnellhefter oder lediglich mit einem
Heftstreifen versehen bei den jeweiligen Betreuerinnen und Betreuern abgegeben
werden. N\"{a}here Details sollten allerdings auch hier mit den zust\"{a}ndigen
Betreuerinnen und Betreuern abgestimmt werden.

Seminararbeiten sollten einen Umfang von ca. 15 Seiten aufweisen. Studienarbeiten
und der auf einen Studierenden entfallende Teil einer Projektarbeit sollten einen
Umfang von 30-50 Seiten, Diplom-, Bachelor- und Masterarbeiten von 50-80 Seiten
vorweisen (Richtgr\"{o}\ss e). Abweichungen von diesen Vorgaben sind mit der Betreuerin
oder dem Betreuer abzustimmen. Alle Arbeiten sind grunds\"{a}tzlich maschinenschriftlich
zu erstellen. F\"{u}r die Schrift ist i.d.R. die Gr\"{o}\ss e 12 pt f\"{u}r Textteile, 10 pt f\"{u}r
Fu\ss noten vorgesehen. Als Schriftart sollten g\"{a}ngige Standardschriften (z.B. Times
New Roman) verwendet werden. Bei der Wahl der Schriftart ist auf ein leicht zu
lesendes Schriftbild zu achten. Ferner sollten bei gegebenen Randeinstellungen
ungef\"{a}hr genau so viele Zeichen pro Zeile gef\"{u}hrt werden wie in der Standardschrift
Times New Roman.

Das Manuskript ist jeweils einseitig auf DIN-A4 (Schreibmaschinen-)Papier zu
beschriften. Freizulassen sind insgesamt 7 cm Rand f\"{u}r Korrekturvermerke, Binde- und
Heftrand. Hierbei sollte der linke Seitenrand etwa 3 cm oder rechte Seitenrand 4 cm
betragen. Am oberen Rand sollten bis zum Text etwa 2,5 cm (bis zur Seitenzahl
mindestens 1 cm) freigelassen werden. Am unteren Rand reichen ca. 2 cm. Das
Seitenformat ist bei s\"{a}mtlichen Arbeiten einheitlich zu gestalten und bildet den
Rahmen f\"{u}r jede Form der Gestaltung des textlichen Erscheinungsbildes (Kopfzeilen,
Fu\ss noten, Graphiken, Tabellen usw.). In den Textteilen ist ein Zeilenabstand von 1,5
einzuhalten. Bei den Fu\ss noten kann ein einzeiliger Zeilenabstand gew\"{a}hlt werden,
dies ist ebenfalls im Literaturverzeichnis m\"{o}glich. Einzelne Abs\"{a}tze sollen sich
deutlich, z. B. durch einen Anfangsabstand und/oder eine Einr\"{u}ckung der ersten
Zeile, voneinander absetzen.

Das Titelblatt ist f\"{u}r alle Typen von wissenschaftlichen Arbeiten dem Titelblatt der
hier vorliegenden Arbeit entsprechend zu gestalten. Aus dem Titelblatt muss
inhaltlich das Bearbeitungsthema, die Art der Arbeit, der Referent/die Referentin,
der Autor/die Autorin der Arbeit sowie das Abgabedatum hervorgehen.

Prinzipiell k\"{o}nnen alle Formen wissenschaftlicher Arbeiten auch in Form von
Gruppenarbeiten zugelassen werden. Hierbei ist es erforderlich, sowohl im
Inhaltsverzeichnis als auch zu Beginn der einzelnen Abschnitte zu vermerken, welche
Kapitel, Textabschnitte und/oder Seiten welchem der Bearbeiter bzw. der
Bearbeiterinnen zuzuordnen sind. Nur somit kann eine differenzierte Begutachtung der
Pr\"{u}fungsleistung \"{u}berhaupt erst erfolgen. Abweichend zu dieser Titelblattgestaltung
werden bei Seminararbeiten zus\"{a}tzlich die Daten der besuchten Veranstaltung
(Seminarname und Semester) mit angegeben.

An das Titelblatt schlie\ss t sich das Inhaltsverzeichnis der Arbeit an, welches einen
\"{U}berblick \"{u}ber dessen Gliederung gibt. Um der Leserin/dem Leser der Arbeit ein
erstes Erfassen der Arbeit erleichtern zu k\"{o}nnen, sind die einzelnen
Gliederungspunkte \"{u}bersichtlich und gro\ss z\"{u}gig (z.B. durch Einr\"{u}ckungen der
jeweiligen Gliederungsebenen) anzuordnen. F\"{u}r jeden einzelnen Gliederungspunkt sind
die entsprechenden Seitenzahlen im Inhaltsverzeichnis anzugeben, die \"{U}berschriften
der einzelnen Textabschnitte m\"{u}ssen mit den jeweiligen Seiten \"{u}bereinstimmen.
Erg\"{a}nzend enth\"{a}lt das Inhaltsverzeichnis auch Hinweise auf das Abk\"{u}rzungs-,
Abbildungs-, Tabellen- und Literaturverzeichnis sowie m\"{o}glicherweise auf einen
Anhang (vgl. Kap. 3.2.2).

\subsubsection{Abk\"{u}rzungen, Tabellen und Abbildungen}

Im laufenden Text der Arbeit sind Abk\"{u}rzungen m\"{o}glichst sparsam zu verwenden, da
durch deren vielfachen Gebrauch das Verst\"{a}ndnis der Arbeit u.U. erheblich
eingeschr\"{a}nkt werden kann. Abk\"{u}rzungen, welche auf reiner Bequemlichkeit beruhen,
sind somit also nicht angebracht, wie z.B. \glqq WM\grqq\ f\"{u}r \glqq
Wissensmanagement\grqq\ oder \glqq MC\grqq\ f\"{u}r \glqq Management Consulting\grqq.
Dagegen sind gel\"{a}ufige Abk\"{u}rzungen des allgemeinen Sprachgebrauchs (wie etc., z. B.,
usw., vgl. - Ma\ss stab ist hier der Duden) selbstverst\"{a}ndlich anerkannt und m\"{u}ssen im
Allgemeinen auch nicht in ein eventuell erstelltes Abk\"{u}rzungsverzeichnis aufgenommen
werden. Allgemein \"{u}bliche Abk\"{u}rzungen, bspw. f\"{u}r Zeitschriftentitel, so wie sie in
der Tabelle 4 beispielhaft dargestellt sind, f\"{u}r Gesetzestexte oder auch f\"{u}r
Organisationen, k\"{o}nnen verwendet werden.

\begin{center}
\begin{table}[h] \centering
\renewcommand{\arraystretch}{1.3}
\begin{tabular}{|l|l|}
\multicolumn{1}{c}{\hspace{5cm}} & \multicolumn{1}{c}{\hspace{5cm}}\\ \hline AER
American Economic Review & DBW Die Betriebswirtschaft \\ \hline JofA Journal of
Accountancy & JPE Journal of Political Economy \\ \hline QJE Quarterly Journal of
Economics & ZfB Zeitschrift f\"{u}r Betriebswirtschaft \\ \hline
\end{tabular}
\singlespacing \caption{Beispiele f\"{u}r Abk\"{u}rzungen von
Fachzeitschriften}\label{AbkJournals} {Quelle: entsprechend aktueller Ausgaben der
genannten Fachzeitschriften}
\renewcommand{\arraystretch}{1.3}
\onehalfspacing
\end{table}
\end{center}

S\"{a}mtliche Tabellen und Abbildungen im Text sollten eine m\"{o}glichst klare Bezeichnung
haben. Daneben ist stets auch ein Quellennachweis zu erbringen. S\"{a}mtliche Tabellen
und Abbildungen sind fortlaufend kapitel- oder textweise mit arabischen Ziffern zu
nummerieren.

Abk\"{u}rzungs- und Abbildungs- bzw. Tabellenverzeichnis erkl\"{a}ren bzw. dokumentieren,
wie die im Text verwendeten Kurzschreibweisen zu verstehen sind bzw. auf welchen
Seiten sich grafische und tabellarische \"{U}bersichten befinden. Das
Abk\"{u}rzungsverzeichnis enth\"{a}lt alphabetisch geordnet die im Text verwendeten
Abk\"{u}rzungen mit ihrer ausf\"{u}hrlichen Bezeichnung. Im Abbildungs- und
Tabellenverzeichnis werden alle im Text verwendeten \"{U}bersichten mit ihrer
Bezeichnung und der betreffenden Seitennummer aufgef\"{u}hrt. Diese Verzeichnisse
geh\"{o}ren - zusammen mit dem Inhaltsverzeichnis - zu den Grundinformationen f\"{u}r den
Umgang mit dem Text und werden daher dem eigentlichen Textteil vorangestellt.

Umfangreicheres und erg\"{a}nzendes Material zur Arbeit, wie sie etwa gr\"{o}\ss ere
tabellarische und grafische Darstellungen, mathematische Berechnungen, statistische
Analysen oder auch l\"{a}ngere Gesetzestexte darstellen k\"{o}nnen, ist in einem Anhang am
Ende der Arbeit unterzubringen. Generell ist zu beachten, dass der Anhang kein \glqq
Auffangbecken\grqq\ f\"{u}r Tabellen und Abbildungen darstellt, die einen unmittelbaren
Teil der inhaltlichen Bearbeitung darstellen. Vielmehr sollte ein Anhang lediglich
zus\"{a}tzliche Informationen zum eigentlichen Text beinhalten, die einem vertiefenden
Textverst\"{a}ndnis f\"{o}rderlich sein k\"{o}nnten. Bei einem entsprechenden Umfang ist dem
Anhang ein eigenst\"{a}ndiges Verzeichnis voranzustellen.

\section{Beratung, Betreuung und Bewertung}\label{Beratung}

Die in diesem Text genannten Format- und Zitiervorgaben stehen h\"{a}ufig bei der Frage
nach den Erfordernissen an die Anfertigung einer wissenschaftlichen Abschlussarbeit
im Vordergrund. Allerdings stellt die Erf\"{u}llung die Vorgaben nur eines von mehreren
Kriterien dar, welche f\"{u}r die Beurteilung einer Abschlussarbeit herangezogen werden
k\"{o}nnen.\footnote{Vgl. dazu auch \cite{spd2004}, S. 126-131.}

Aus den bisherigen Ausf\"{u}hrungen dieser Arbeit k\"{o}nnen somit f\"{u}r eine
wissenschaftliche Abschlussarbeit die folgenden Bewertungsfelder abgeleitet werden.

\begin{itemize}

\item Die Entwicklung einer Fragestellung sowie eines Untersuchungsaufbaus bildet
stets den Ausgangspunkt der eigenen Bearbeitung. F\"{u}r die Entwicklung der
Fragestellung ist eine systematische Problemanalyse erforderlich. Darin wird die
Relevanz der Zielsetzung aus der Bedeutung des zu behandelnden Problems im
Themengebiet abgeleitet. Im Untersuchungsaufbau ist insbesondere der gew\"{a}hlte
Untersuchungsansatz sowie die Strukturierung der Arbeit zu begr\"{u}nden.

\item In der Literaturauswertung wird eine inhaltliche Aufbereitung sowie eine
problemorientierte Einordnung der f\"{u}r die Bearbeitung relevanten Literatur
vorgenommen. In diesem Analyseschritt sind themenrelevante Begriffe und Konzepte
auszuf\"{u}hren sowie mit Blick auf die eigene Problemstellung kritisch zu reflektieren.

\item Die \glqq Logik\grqq\ der Argumentationsf\"{u}hrung bezieht sich auf die
problemorientierte Synthese der inhaltlichen Ergebnisse. Die Folgerungen m\"{u}ssen
stichhaltig und die einzelnen Argumentationsschritte in der \glqq logischen\grqq\
Klammer, die von der Einleitung zum Ergebnisteil einer Arbeit f\"{u}hrt, nachvollziehbar
sein.

\item Mittels der formalen Gestaltung der Arbeit, und dabei insbesondere auch der
Zitation verwendeter Quellen, wird ein Ausdruck einer sorgf\"{a}ltigen inhaltlichen
Bearbeitung der Themenstellung vermittelt.

\end{itemize}

Die individuelle Bearbeitung eines Themas hat sich an diesem Bewertungskanon zu
orientieren. Das bedeutet auch, dass am Anfang eines solchen Bearbeitungsprozesses
weder alle Probleme aufgeworfen, noch alle Fragen beantwortet werden k\"{o}nnen. Zudem
sind wissenschaftliche Arbeitsprozesse, von der Problemstellung \"{u}ber erste
Gliederungsentw\"{u}rfe bis zur endg\"{u}ltigen Fertigstellung, auch stets durch einen
Austausch und eine Kooperation mit Kommilitoninnen und Kommilitonen sowie einer
Kommunikation mit dem oder den Betreuern/-innen gekennzeichnet.


\newpage
\begin{thebibliography}{}

\singlespacing
\bibitem[\protect\citeauthoryear{Alberth}{\emph{Alberth}}{1998}]{alb1998}{Alberth, M.R. (1998):} \emph{Kurze Gedanken zum wissenschaftlichen Zitieren des Internets},
In: ZfB - Zeitschrift f\"{u}r Betriebswirtschaft, 68. Jg., Heft 12, S. 1367-1374.

\bibitem[\protect\citeauthoryear{B\"{a}nsch}{\emph{B\"{a}nsch}}{1998}]{bae1996}{B\"{a}nsch, A. (1996):} Wissenschaftliches Arbeiten. Seminar- und Diplomarbeiten, M\"{u}nchen
und Wien, 5. Auflage.

\bibitem[\protect\citeauthoryear{Brink}{\emph{Brink}}{2005}]{bri2005}{Brink, A. (2005):} Anfertigung wissenschaftlicher Arbeiten,
M\"{u}nchen und Wien, 2. Auflage.

\bibitem[\protect\citeauthoryear{Corsten}{\emph{Corsten}}{1996}]{cor1996}{Corsten, H. (1996):} \emph{Literatur\"{u}berblick Wissenschaftliches Arbeiten},  In: WiSt
- Wirtschaftswissenschaftliches Studium. 25. Jg., Heft 11. S. 597-600.

\bibitem[\protect\citeauthoryear{Disterer}{\emph{Disterer}}{1998}]{dis1998}{Disterer, G. (1998):} Studienarbeiten schreiben. Diplom-, Seminar- und Hausarbeiten
in den Wirtschaftswissenschaften, Berlin u.a.

\bibitem[\protect\citeauthoryear{Eco}{\emph{Eco}}{1993}]{eco1993}{Eco, U. (1993):} Wie man eine wissenschaftliche Abschlu\ss arbeit schreibt, Heidelberg,
6. Auflage.

\bibitem[\protect\citeauthoryear{Franck}{\emph{Franck}}{2004}]{fra2004}{Franck, N. (2004):} Handbuch Wissenschaftliches Arbeiten, Frankfurt a. M.

\bibitem[\protect\citeauthoryear{Haefner}{\emph{Haefner}}{2000}]{hae2000}{Haefner, K. (2000):} Gewinnung und Darstellung wissenschaftlicher Erkenntnisse,
M\"{u}nchen und Wien.


\bibitem[\protect\citeauthoryear{Jaros-Sturhahn und Schachtner}{\emph{Jaros-Sturhahn und Schachtner}}{1996}]{jss1996}{Jaros-Sturhahn, A. und Schachtner, K. (1996):} \emph{Literaturrecherche im World Wide
Web}, In: WiSt - Wirtschaftswissenschaftliches Studium. 25. Jg., Heft 8. S. 419-422.

\bibitem[\protect\citeauthoryear{Koeder}{\emph{Koeder}}{1988}]{koe1988}{Koeder, K.W. (1988):} \emph{Arbeitsmethodik im Studium. Wissensaufnahme und Verarbeitung},
In: WiSt - Wirtschaftswissenschaftliches Studium. 17. Jg., Heft 1. S. 43-46.

\bibitem[\protect\citeauthoryear{Ridder}{\emph{Ridder}}{2001}]{rid2001}{Ridder, H.-G. (2001):} Wie man eine wissenschaftliche Abschlussarbeit
schreibt, Institut f\"{u}r Betriebsforschung, Universit\"{a}t Hannover.

\bibitem[\protect\citeauthoryear{Rossig und Pr\"{a}tsch}{\emph{Rossig und Pr\"{a}tsch}}{2005}]{rop2005}{Rossig, W. und Pr\"{a}tsch, J. (2005):} Wissenschaftliche Arbeiten, Hamburg., 5.
Auflage.

\bibitem[\protect\citeauthoryear{Schlep\"{u}tz}{\emph{Schlep\"{u}tz}}{2003}]{sch2003}{Schlep\"{u}tz, V. (2003):} \emph{Anleitung zum Lesen eines wirtschaftswissenschaftlichen
Journal-Beitrags}, In: WiSt - Wirtschaftswissenschaftliches Studium. 32. Jg., Heft
5. S. 305-310.

\bibitem[\protect\citeauthoryear{Spoun und Domnik}{\emph{Spoun und Domnik}}{2004}]{spd2004}{Spoun, S. und Domnik, D. (2004):} Erfolgreich studieren. Ein Handbuch f\"{u}r
Wirt-schafts- und Sozialwissenschaftler, M\"{u}nchen.

\bibitem[\protect\citeauthoryear{Theisen}{\emph{Theisen}}{1993}]{the1993}{Theisen, M.R. (1993):} Wissenschaftliches Arbeiten. Technik -
Methodik - Form, M\"{u}nchen, 7. Auflage.

\end{thebibliography}

\newpage
\section*{Anhang}


Anhang 1: Ehrenw\"{o}rtliche Erkl\"{a}rung f\"{u}r Diplom-/Bachelor-/Masterarbeiten



E R K L \"{A} R U N G

Hiermit versichere ich, dass ich die vorliegende Arbeit selbst\"{a}ndig verfasst und
keine anderen als die angegebenen Quellen und Hilfsmittel benutzt habe, dass alle
Stellen der Arbeit, die w\"{o}rtlich oder sinngem\"{a}\ss  aus anderen Quellen \"{u}bernommen
wurden, als solche kenntlich gemacht sind, und dass die Arbeit in gleicher oder
\"{a}hnlicher Form noch keiner Pr\"{u}fungsbeh\"{o}rde vorgelegt wurde.

Ort, Datum Unterschrift der Kandidatin/des Kandidaten


\addcontentsline{toc}{section}{Literaturverzeichnis}
\addcontentsline{toc}{section}{Anhang}



\end{document}